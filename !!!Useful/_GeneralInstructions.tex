\documentclass[xcolor={dvipsnames},t]{beamer}
\usepackage{caption}
\usepackage[utf8]{inputenc}
\usepackage[english,greek]{babel}

\usetheme{Madrid}
\usecolortheme{default}
\setbeamertemplate{enumerate items}[default]
\setbeamercolor*{structure}{bg=white,fg=blue}

% \setbeamercolor*{palette primary}{use=structure,fg=white,bg=structure.fg}
% \setbeamercolor*{palette secondary}{use=structure,fg=white,bg=structure.fg!75}
% \setbeamercolor*{palette tertiary}{use=structure,fg=white,bg=structure.fg!50!black}
% \setbeamercolor*{palette quaternary}{fg=white,bg=black}

% \setbeamercolor{section in toc}{fg=black,bg=white}
% \setbeamercolor{alerted text}{use=structure,fg=structure.fg!50!black!80!black}
% \setbeamercolor{frametitle}{bg=black,fg=white}

% \setbeamercolor{titlelike}{parent=palette primary,fg=structure.fg!50!black}
% \setbeamercolor{frametitle}{bg=gray!10!white,fg=black}

% \setbeamercolor*{titlelike}{parent=palette primary}

% \setbeamercolor{block title example}{bg=black,fg=white}

% \usepackage{times,url}

% \setbeamertemplate{footline}[frame number]
% \setbeamertemplate{footline}[frame number]{}

\setbeamertemplate{footline}{}

\setbeamertemplate{footline}
% {
%   \leavevmode%
%   \hbox{%
%   \begin{beamercolorbox}[wd=.333333\paperwidth,ht=2.25ex,dp=1ex,center]{author in head/foot}%
%     \usebeamerfont{author in head/foot}\insertsection
%   \end{beamercolorbox}%
%   \begin{beamercolorbox}[wd=.333333\paperwidth,ht=2.25ex,dp=1ex,center]{title in head/foot}%
%     \usebeamerfont{title in head/foot}\insertsubsection
%   \end{beamercolorbox}%
%   \begin{beamercolorbox}[wd=.333333\paperwidth,ht=2.25ex,dp=1ex,right]{date in head/foot}%
%     \usebeamerfont{date in head/foot}\insertshortdate{}\hspace*{2em}
%     \insertframenumber{} / \inserttotalframenumber\hspace*{2ex} 
%   \end{beamercolorbox}}%
%   \vskip0pt%
% }

%------------------------------------------------------------
%This block of code defines the information to appear in the
%Title page
\title[Οδηγιες]{{\small Ηλεκτρικα Κυκλωματα Ι}\\Οδηγιες για Εθελοντες Ασκησεων}
\author[Αθ. Κεχαγιας]{Αθ. Κεχαγιας}


\date[2024]{2024}

%End of title page configuration block
%------------------------------------------------------------


\begin{document}

%The next statement creates the title page.
\frame{\titlepage}



%---------------------------------------------------------


\section{Οδηγιες για Εθελοντες Ασκησεων}

%---------------------------------------------------------
\begin{frame}
\frametitle{Γενικες Αρχες}

\begin{enumerate}
\pause\item Αυτο ειναι για να μαθετε εσεις και να μελετησουν οι αλλοι.

\pause\item Αρα καντε ενα καλο δειγμα απο ολα τα ειδη ασκησεων.

\pause (Υπαρχει τεραστιος αριθμος ασκησεων, δεν θα τις κανετε ολες.)

\pause\item Χρησιμοποιειστε ομαδες δυο ατομων για συνεργασια, συζητηση κτλ.

\pause\item Αναθεστε την ιδια ασκηση σε τουλαχιστον δυο ομαδες για συγκριση αποτελεσματων.

\pause (Βεβαια θα κανετε επαληθευση με το {\latintext  Elab}.)

\pause\item Δωστε αρκετα λογια για να γινετε κατανοητη η λυση αλλα οχι παρα πολλα. 

\end{enumerate}

\end{frame}
%---------------------------------------------------------
\begin{frame}
\frametitle{Περιεχομενα Λυσεων}

\pause Αυτα που ζηταει η εκφωνηση και επιπλεον:

\begin{enumerate}

\pause\item Την {\latintext  netlist}.
\pause\item Τους πινακες προσπτωσης και βροχων (και το αντιστοιχο θεμελιωδες συνολο βροχων). 
\pause\item Πληρη λυση του κυκλωματος ($\mathbf{v,i,u}$).
\pause\item Τις εξισωσεις κομβων και κλαδων απο τις $A\mathbf{i}=\mathbf{0}$, 
$A^T\mathbf{v}=\mathbf{u}$.

\end{enumerate}

\end{frame}
%---------------------------------------------------------
\begin{frame}
\frametitle{Διευκρινισεις}

\begin{enumerate}

\pause\item Το {\latintext  Elab} δινει πληρη λυση του κυκλωματος.

\pause\item Αν υπαρχουν ποσοτητες (π.χ. καποια απροσδιοριστη $R$) λυστε με αυτες ως παραμετρους.  
(Βοηθεια:  {\latintext  Symbolic Toolbox}.)

\pause\item Στα σχηματικα: οι ρομβοι ειναι εξαρτημενες πηγες (τασης ή ρευματος)

\pause\item Το βολτομετρο συνδεεται σε σειρα, εχει μηδενικη αντισταση.

\pause\item Το αμπερομετρο συνδεεται παραλληλα, εχει απειρη αντισταση.

\end{enumerate}

\end{frame}
%---------------------------------------------------------
\begin{frame}
\frametitle{Παραδοτεα}

\begin{enumerate}

\pause\item Στο {\latintext  github}.
\pause\item Ενα αρχειο {\latintext  Live Script (*.mlx)} ανα ασκηση μαζι με: 

\begin{enumerate}
\pause\item Κωδικα {\latintext  Matlab/ELab} και συνοδευτικα αρχεια. 
\pause\item Σχηματικα (απο {\latintext  Falstad} ή {\latintext  Qucs}) σε μορφη 
{\latintext  jpg} και {\latintext pdf}.

\pause (Αργοτερα θα σας δειξω κατι πολυ καλυτερο για σχηματικα.)

\end{enumerate}
\end{enumerate}

\end{frame}
%---------------------------------------------------------
\begin{frame}
\frametitle{{\latintext  Matlab}}

\begin{enumerate}
\pause\item  {\latintext  Symbolic Toolbox}
\pause\item  {\latintext  ELab}
\begin{enumerate}
\pause\item Ενημερωση του {\latintext  path}. 
\pause\item Υπαρχει ενα κολπο του {\latintext  ELab} για εξαρτημενες πηγες ρευματος.
\end{enumerate}
\end{enumerate}

\end{frame}
%---------------------------------------------------------
\begin{frame}
\frametitle{Οι Ασκησεις !!!}

\begin{enumerate}

\pause\item  \textbf{{\latintext  SD03:}}
3.2.2-4, 11-26,
3.3.1-2, 7, 10-12, 14-17,
3.4.1-2, 4, 7-15, 17-18,
3.5.1-3,
3.6.1-47,
3.7.1-4.

\pause\item  \textbf{{\latintext  SD04:}}
4.2.1-8,
4.3.1-14,
4.4.1-22,
4.5.1-6,
4.6.1-15,
4.7.1-4, 6-16,
4.9.1-10.

\end{enumerate}

\end{frame}
%---------------------------------------------------------
\begin{frame}
\frametitle{{\latintext  ToDo}}

\begin{enumerate}
\pause\item  Αρχιστε το.
\pause\item  Στειλτε μου εναν υπευθυνο ανα ομαδα.
\pause\item  Καθε ομαδα: να στελνετε μια (π.χ. εβδομαδιαια) ενημερωση.
\pause\item  Θα σας στελνω σφαιρικη ενημερωση και περαιτερω οδηγιες. 
\pause\item  Σε δυο εβδομαδες ξανα συναντηση. 
\end{enumerate}

\end{frame}
%---------------------------------------------------------
\begin{frame}
\end{frame}
%------------------------------------------------------------------
\end{document}



 